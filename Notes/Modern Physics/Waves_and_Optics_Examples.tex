\documentclass[a4paper,10pt]{report}
\usepackage[utf8]{inputenc}
\usepackage{amssymb}
\usepackage{amsthm}
\usepackage{amsmath}
\usepackage{graphicx}

\setlength{\parindent}{0pt}

\begin{document}
\title{Examples: Modern Physics - Waves, Optics}
\author{Zack Garza}
\maketitle
\tableofcontents

\chapter{Light as Waves}
\begin{enumerate}
	\item
		A thin film of oil ($n$ = 1.28) is located on smooth, wet pavement. When viewed perpendicularly, the film reflects most strongly red light at 640 nm and reflects no green light at 512 nm.

		What is the minimum thickness of the film?

		\textit{Hints:}

		There is a bottom interface!

		Be careful with the algebra on this one.

		\hrulefill

		\textit{Solution:}

		\begin{align*}
			2t &= \frac{m\lambda_{\text{const}}}{n} \\
			2t &= \frac{(m+\frac{1}{2})\lambda_{\text{dest}}}{n} \\
			\frac{\lambda_1}{\lambda_2} &= 1 + \frac{1}{2m} \\
			m &= 2 \\
			t &= 500 \times 10^{-9} \text{m}
		\end{align*}

		\hrulefill

	\item
		An air wedge is formed between two glass plates separated at one edge by a very fine wire of circular cross section as shown in the figure below. When the wedge is illuminated from above by 670 nm light and viewed from above, 25 dark fringes are observed. Calculate the diameter $d$ of the wire.

		\textit{Hints:}

		Only consider the light that interacts with the top portion of the wedge.

		What are the shifts at the interfaces?

		Assume the 25 fringes are viewed symmetrically about some central point.

		\textit{Solution:}

		For constructive interference when $n_1 < n_2 > n_3$:
		\begin{align*}
			2t = m\lambda
		\end{align*}
		and $m$ = 24, so $d$ = 8.02 $\mu$m.

\end{enumerate}
\end{document}