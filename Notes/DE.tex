\documentclass[a4paper,10pt]{report}
\usepackage[utf8]{inputenc}
\usepackage{amssymb}
\usepackage{amsthm}
\usepackage{amsmath}
\setlength{\parindent}{0pt}

\begin{document}
\title{Differential Equations and Linear Algebra Spring 2014 Notes}
\author{Zack Garza}
\maketitle
\tableofcontents

\chapter{Vector Spaces}
Next Exam: April 2nd. Covers 4.6$\rightarrow$5.1.

\section{Bases}
\subsection{Determining a Basis}
A set $S$ that forms a basis for a vector space $V$ must satisfy two conditions:

\begin{enumerate}
 \item $S$ is set of linearly independent vectors.
 \item $S$ spans $V$.
\end{enumerate}

\textbf{Does a set $S$ form a basis for a vector space $V$?}

First, check for linear independence. If dim[$V$]=$n$ and $S$ contains $n$ linearly independent vectors, $S$ is guaranteed to form a basis for $V$.

Note: dim[$P_n$]$=n+1$.

\textbf{Example 1}

Determine a basis for
\begin{align*}
 S = \left\{a_0 + a_1 x + a_2 x^2\mid a_0,a_1,a_2 \in \mathbb{R} \land a_0 - a_1 -2a_2 =0\right\}.
\end{align*}

Let $a_2=t, a_1=s, a_0=s+2t$, then
\begin{align*}
 S 	&= \left\{ (s+2t) + (sx+tx^2)\mid s,t\in\mathbb{R} \right\} \\
	&= \left\{ (s+sx) + (2t+tx^2)\mid s,t\in\mathbb{R} \right\} \\
	&= \left\{ s(1+x) + t(2+x^2)\mid s,t\in\mathbb{R} \right\} \\
	&= \text{span}\left\{(1+x),(2+x^2)\right\}
\end{align*}
and a basis for $S$ is
\begin{align*}
 \left\{(1+x), (2+x^2)\right\}
\end{align*}




\section{Inner Product Spaces (4.11)}
\textit{March 24, 2014}\\

\subsection{Axioms}
\noindent4 Axioms of an Inner Product
\begin{enumerate}
  \item $V_1 \cdot V_1 >= 0$ and $V_1\cdot V_1$ iff $V_1 = 0$

    Check that the scalar result is positive or zero.\\
    Show that $\langle A,A\rangle = 0$ forces the coefficients to be zero.
  \item $V_1 \cdot V_2 = V_2 \cdot V_1$
  \item $(cV_1)\cdot V_2 = c(V_1\cdot V_2)$
  \item $V_1 \cdot (V_2 + V_3) = V_1 V_2 + V_1 V_3$
\end{enumerate}

\subsection{Orthogonality}
$\langle p,q\rangle =0 \Rightarrow$ Orthogonality.

\subsection{The Gram-Schmidt Procedure}
Given a set of vectors
\begin{align*}
 S = \left\{\mathbf{v_1, v_2, \cdots v_n}\right\},
\end{align*}
the Gram-Schmidt procedure produces a corresponding orthogonal set
\begin{align*}
 S' = \left\{\mathbf{u_1, u_2, \cdots u_n}\right\}
\end{align*}
the is a basis for the same vector space as $S$.

Given the set $S$, $S'$ is found using the following pattern:
\begin{align*}
 \mathbf{u_1} &= \mathbf{v_1} \\
 \mathbf{u_2} &= \mathbf{v_2} - \text{proj}_{\mathbf{u_1}} \mathbf{v_2}\\
 \mathbf{u_3} &= \mathbf{v_3} - \text{proj}_{\mathbf{u_1}} \mathbf{v_3} - \text{proj}_{\mathbf{u_2}} \mathbf{v_3}\\
\end{align*}
where
\begin{align*}
 \text{proj}_{\mathbf{u}} \mathbf{v} = (\text{scal}_{\mathbf{u}} \mathbf{v})\frac{\mathbf{u}}{\mathbf{u}}
\end{align*}



A few definitions are needed:
\begin{align*}
 \text{scal}_{u1}
\end{align*}


\subsection{Examples}
\subsubsection*{1.}
Let $A, B. C \in M_2 (\mathbb{R})$. Define $\langle A,B\rangle = a_{11}b_{11}+2a_{12}b{12}+3a_{21}b_{21}$. Does this define an inner product on $M_2 (\mathbb{R})$?

\subsubsection{2.}
Instead, let $\langle A,B\rangle = a_{11} + b_{22}$. Does this define an inner product on $M_2(\mathbb{R})$?

\subsubsection{3.}
Let $p=a_0 + a_1 x + a_2 x^2$ and $q=b_0 + b_1 x + b_2 x^2$.

Define $\langle p,q\rangle = \sum_{i=0}^{2}(i+1)a_i b_i$. Does this define an inner product on $P_2$?

\subsubsection{4.}
Let $f,g \in C((-\infty, \infty))$. Define

\begin{equation*}
\langle f,g\rangle = \int_a^b f(x)g(x)dx.
\end{equation*}

\section{Gram}

\end{document}
