\documentclass[twocolumn,english]{IEEEtran}
\usepackage[T1]{fontenc}
\usepackage{babel}
\usepackage{amsthm}
\usepackage{amsmath}
\usepackage{graphicx}
\usepackage[unicode=true,
bookmarks=true,bookmarksnumbered=true,bookmarksopen=true,bookmarksopenlevel=1,
breaklinks=false,pdfborder={0 0 0},backref=false,colorlinks=false]
{hyperref}
\usepackage{bm}
\usepackage{amsmath}
\usepackage{amssymb}
\usepackage{natbib}
\usepackage{array}
\usepackage{calc}
\newcolumntype{W}{>{\centering\arraybackslash}m{25mm}}
\newcolumntype{L}{>{\centering\arraybackslash}m{15mm}}
\makeatletter
\setcounter{topnumber}{2}
\setcounter{bottomnumber}{2}
\setcounter{totalnumber}{4}
\renewcommand{\topfraction}{0.85}
\renewcommand{\bottomfraction}{0.85}
\renewcommand{\textfraction}{0.15}
\renewcommand{\floatpagefraction}{0.7}
\usepackage{float}
\usepackage{booktabs}

\hypersetup{
pdftitle=  {Lab 4: Diode Circuits},
pdfauthor= {Alex Landry, Nolan Guettler, Zack Garza}%TODO: Fill in names
pdfpagelayout=OneColumn, pdfnewwindow=true, pdfstartview=XYZ, plainpages=false}

\begin{document}

\title{Lab 4: Diode Circuits}
\author{Alex Landry, Nolan Guettler, Zack Garza} %TODO: Fill in names
\IEEEspecialpapernotice{Engineering 17L \\ Effective Date of Report: \today}

\markboth{Lab 4: Diode Circuits}{Zack Garza}
\maketitle

\begin{abstract}
\IEEEPARstart{D}{iodes} are crucial elements in modern circuitry. They work like one-way gates for current, and have a wide array of electrical applications. This report investigates several circuits which utilize both diodes and zener diodes,  which allow one-way current travel until a certain maximum is reached. Several circuits that utilize these diodes are examined, including clipping circuits, clamping circuits, voltage regulator circuits, AC to DC conversion circuits, and circuits that implement digital logic.
\end{abstract}

\tableofcontents

\section{Equipment List}
\begin{enumerate}
  \item Breadboard
  \item \underline {Measuring Equipment:}
    \begin{enumerate}
      \item Digital Oscilloscope
      \item Digital Multi-Meter \hfill[x2]
    \end{enumerate}
  \item \underline {Power Sources:}
    \begin{enumerate}
    \item DC Power Supply
    \item Function Generator
    \end{enumerate}
  \item \underline{Resistors:}
    \begin{enumerate}
    \item 1.0 k$\Omega$ (1/4 W) \hfill[x1]
    \item 10.0 k$\Omega$ (1/2 W) \hfill[x1]
    \item Variable Resistor Box
    \end{enumerate}
  \item \underline{Diodes:}
    \begin{enumerate}
     \item 1N4002 Diode \hfill[x2]
     \item 1N4737 Zener Diode \hfill[x1]
    \end{enumerate}
  \item \underline{Capacitors:}
    \begin{enumerate}
     \item 0.1 $\mu$F \hfill[x1]
     \item 1.0 $\mu$F (Electrolytic) \hfill[x1]
    \end{enumerate}
\end{enumerate}

\section{Methodology}
\subsection{Diodes as Clippers}
\begin{figure}[h]
  \begin{centering}
  \begin{center}
  \includegraphics[width=\linewidth]{./1.png}
  \label{fig:circuit_1a}
  \caption{Single Diode Clipping Circuit Diagram}
  \end{center}
  \par\end{centering}
  \end{figure}

  \begin{figure}[h]
  \begin{centering}
  \begin{center}
  \includegraphics[width=\linewidth]{./2.png}
  \label{fig:circuit_1b}
  \caption{Parallel Diode Clipping Circuit Diagram}
  \end{center}
  \par\end{centering}
  \end{figure}

  \begin{enumerate}
    \item The circuit was built on the protoboard with $R_1=10$ k$\Omega$. %TODO: Need ref to diagram
    \item The oscilloscope was set to DC coupling mode, and Channels 1 and 2 were displayed simultaneously.
    \item A \textbf{1 kHz triangle wave} was used as the input signal. $V_{s1}$ and $V_{\text{Out 1}}$ were monitored on Channels 1 and 2 respectively.
    \item The following parameters of the input signal were varied, and the effects recorded:
      \begin{enumerate}
       \item DC Offset
       \item Frequency
       \item Peak-to-Peak Voltage
      \end{enumerate}
    \item Screen captures were generated that indicated the clipping behavior of the circuit.
    \item The second diode was wired parallel to the first to observe the resulting clipping effects. %TODO: Explain with diagram, labels?
    \item The following parameters of the input signal were varied, and the effects recorded:
      \begin{enumerate}
       \item DC Offset
       \item Amplitude
       \item Waveform Type
      \end{enumerate}
  \end{enumerate}

\subsection{Diode Clamping}
\begin{figure}[h]
  \begin{centering}
  \begin{center}
  \includegraphics[width=\linewidth]{./3a.png}
  \label{fig:circuit_2}
  \caption{Clamping Circuit Diagram}
  \end{center}
  \par\end{centering}
  \end{figure}

  \begin{enumerate}
   \item The circuit was built on the protoboard with $R_2 = 10$ k$\Omega$. %TODO: Ref to diagram
   \item A \textbf{1kHz sine wave} input signal was used, with the \textbf{DC Offset} set to \textbf{zero}.
   \item The oscilloscope was set to DC coupling mode. %TODO: Explain why DC coupling in results
   \item The oscilloscope was then wired to monitor $V_{s2}$ and $V_{\text{Out 2}}$.
   \item The input and output signals were measured on Channels 1 and 2 respectively. %TODO: Ref to diagram
   \item The DC Offset of the input signal was varied, and the effects on the output signal were measured and recorded.
  \end{enumerate}

\subsection{Voltage Regulator}
\begin{figure}[h]
  \begin{centering}
  \begin{center}
  \includegraphics[width=\linewidth]{./3.png}
  \label{fig:circuit_3}
  \caption{Voltage Regulator Circuit Diagram}
  \end{center}
  \par\end{centering}
  \end{figure}

  \begin{enumerate}
   \item The circuit was built on the protoboard. %TODO: Ref to diagram, add exact values?
   \item A 20 V DC power generator was used as the source voltage.
   \item The oscilloscope was wired to measure the load variables $V_L$ and $i_L$.
   \item The resistance $R_L$ was varied, and data points were collected for both $V_L$ and $i_L$.
   \item A plot was generated of the data points as the experiment was conducted in order to determine where more data points were needed.
  \end{enumerate}

\subsection{AC $\rightarrow$ DC Converter} %Rightarrow causes error, but compiles. Only affects autogenerated table of contents in PDF.
\begin{figure}[h!]
  \begin{centering}
  \begin{center}
  \includegraphics[width=\linewidth]{./acdc_circuit_diag.png}
  \label{fig:acdc_circuit_diag}
  \caption{DC Power Supply Circuit Diagram}
  \end{center}
  \par\end{centering}
  \end{figure}

  %TODO: Circuit diagram
  \begin{enumerate}
    \item The circuit was built on the protoboard. %TODO: Ref to diagram.
    \item The function generator was set to produce a \textbf{1 kHz sine wave}.
    \item The capacitor was tested for its polarization and connected appropriately.
    \item The oscilloscope was wired to measure $V_s$ and $V_{\text{Out}}$ on Channels 1 and 2 respectively.
    \item The oscilloscope was set to DC coupling on both channels.
    \item The oscilloscope was zeroed before capturing data. %TODO: Recalibrate? Or adjust levels?
    \item The resistance $R_L$ was varied, and the results were recorded.
    \item The voltages were measured with DVMs and recorded. %TODO: Which voltages?
  \end{enumerate}

\subsection{Diode Logic Circuits}
\begin{figure}[h!]
  \begin{centering}
  \begin{center}
  \includegraphics[width=\linewidth]{./and_gate.png}
  \label{fig:and_gate_diagram}
  \caption{Logical AND Gate Circuit Diagram}
  \end{center}
  \par\end{centering}
  \end{figure}

  \begin{figure}[h!]
  \begin{centering}
  \begin{center}
  \includegraphics[width=\linewidth]{./or_gate.png}
  \label{fig:or_gate_diagram}
  \caption{Logical OR Gate Circuit Diagram}
  \end{center}
  \par\end{centering}
  \end{figure}

  \begin{enumerate}
   \item The circuit for the \textbf{AND} gate was built on the protoboard.
   \item A \textbf{5 V DC} power source was wired into the protoboard.
   \item The oscilloscope was wired to measure $V_{\text{Out}}$.
   \item Different combinations of voltages were applied to the inputs, and the outputs were recorded.
   \item The procedure was repeated for an \textbf{OR} gate.
  \end{enumerate}

\noindent\hrulefill


\section{Data and Analysis}
\subsection{\textbf{Clipping Circuit}}

\begin{figure}[h]
  \begin{centering}
  \begin{center}
  \includegraphics[width=\linewidth]{./Circuit_1.png}
  \caption{Source and diode voltage overlayed to show clipping behavior. The flat portion waveform just above the x-axis represents the clipped output voltage, while the triangle wave was used as the circuit's source signal.}
  \label{fig:circuit_1_results}
  \end{center}
  \par\end{centering}
\end{figure}

As the parameters outlined in the Methodology section were varied, the following observations were made, resulting in the waveforms shown in Figure~\ref{fig:circuit_1_results}.

\begin{enumerate}

  \item \textit{DC Offset}

    Changing the DC Offset of the signal generator changes the magnitude of the voltage supplied, but does not have a discernible effect on the output voltage. The diode effectively cut off the signal at approximately zero volts, regardless of what voltage was supplied by the source. \\

  \item \textit{Frequency}

    Changing the frequency of the source signal changes the edge behavior of the output voltage slightly.
    At low frequencies, the waveform of the output signal will mirror the input signal, with all voltages above zero rounded down to zero, which effectively flattens out the peaks.
    At higher frequencies, the source and output voltages begin to slip out of phase, and the waveform becomes sinusoidal with a peak at zero volts. \\

  \item \textit{Signal Amplitude}

    Increasing the amplitude of the source signal increases the steepness of the waveform of the output voltage, but does not have a large effect on the clipping behavior.
\end{enumerate}
It was found that the circuit deviated slightly from an ideal clipping circuit. While the theoretical maximum output voltage should be clipped to zero volts, the actual maximum output voltage was measured to be \textbf{552 mV}, which is shown in Figure~\ref{fig:cutoff_voltage}.
\begin{figure}[h]
  \begin{centering}
  \begin{center}
  \includegraphics[width=\linewidth]{./Cutoff_Voltage.png}
  \caption{This is a zoomed in portion of the maximum clipped voltage, centered about zero. The deviation from zero volts was measured to be positive 552 mV, showing that the actual circuit behavior differs slightly from its theoretical ideal cutoff voltage.}
  \label{fig:cutoff_voltage}
  \end{center}
  \par\end{centering}
  \end{figure}





\subsection{\textbf{Clamping Circuit}}
\begin{figure}[h!]
  \begin{centering}
  \begin{center}
  \includegraphics[width=\linewidth]{./clamping_results.png}
  \label{fig:clamping_results}
  \caption{Clamping Circuit}
  \end{center}
  \par\end{centering}
  \end{figure}


\subsection{\textbf{Voltage Regulator}}

\begin{figure}[h]
  \begin{centering}
  \begin{center}
  \includegraphics[width=\linewidth]{./voltage_reg.png}
  \caption{Plots of the load resistor's current and voltage, and its deviation from the behavior of an ideal voltage source.}
  \label{fig:voltage_reg}
  \end{center}
  \par\end{centering}
  \end{figure}






\subsection{\textbf{AC $\rightarrow$ DC Converter}}
This next circuit uses diodes and capacitors to level out an AC sine wave to approximate a DC source. First the diode connected to the positive terminal of the power supply makes the current only move in one direction, creating times of source current and times of no source current. During the time of source current the 1$\mu$F capacitor stores voltage. Then during the time of no source current, the capacitor discharges, creating a more level voltage source. The final component to this circuit is the zener diode.

The zener diode is like a pressure valve: if the voltage gets too high, the diode breaks down. This keeps the voltage across the load constant approximately around its breakdown voltage.

When the load resistance is lowered, the voltage across the Load decreases. The voltage across the zener diode connected in parallel with the load also decreases. If this voltage drops below the breakdown voltage, the voltage across the load will not be the breakdown voltage.

\begin{figure}[h!]
  \begin{centering}
  \begin{center}
  \includegraphics[width=\linewidth]{./dcp.png}
  \label{fig:dcp}
  \caption{DC Voltage}
  \end{center}
  \par\end{centering}
  \end{figure}

When we increase the frequency of the source to 1 kHz, the output voltage levels out to the breakdown voltage of the zener diode. The voltage across the load using an ordinary DC volt meter gives us 7.12 V, while the oscilloscope measures the mean voltage across the load as 7.20 V. The supplied data sheet gives a breakdown voltage of around 7.5 volts.

The value the DC voltmeter gives us is \textbf{5\%} different then the data sheet and \textbf{1\%} different then the oscilloscope mean value. This may just be due to the internal resistance of the meter.

\begin{figure}[h]
  \begin{centering}
  \begin{center}
  \includegraphics[width=\linewidth]{./DC1.png}
  \caption{Display of input and output signals. The output voltage is not strictly constant, as it exhibits an exponential voltage decay due to the capacitor.}
  \label{fig:DC1}
  \end{center}
  \par\end{centering}
\end{figure}

\begin{figure}[h]
  \begin{centering}
  \begin{center}
  \includegraphics[width=\linewidth]{./DC2.png}
  \caption{As the frequency increases, the voltage drop between cycles decreases.}
  \label{fig:DC2}
  \end{center}
  \par\end{centering}
\end{figure}

\begin{figure}[h]
  \begin{centering}
  \begin{center}
  \includegraphics[width=\linewidth]{./DC3_5k.png}
  \caption{Zooming out at a high frequency shows that the output closely approximates a constant DC voltage.}
  \label{fig:DC3}
  \end{center}
  \par\end{centering}
\end{figure}

\begin{figure}[h]
  \begin{centering}
  \begin{center}
  \includegraphics[width=\linewidth]{./DC4_1k.png}
  \caption{Changing the load resistance directly affects the voltage delivered to it. Decreasing the resistance by a factor of 5 lowered the output voltage by a factor of 3.}
  \label{fig:DC4}
  \end{center}
  \par\end{centering}
\end{figure}





\subsection{\textbf{Logic Gates}}
The final circuits investigated were diode logic circuits.
These circuits operate by utilizing diodes to produce an output voltage that is dependent on two input voltages, A and B.
Depending how the voltage sources and diodes are oriented, the output voltage can be programmed to hold different voltages for the different voltage combinations of A and B.
The input-output patterns of the gates are essential in computer circuitry, since they can be used to mimic the boolean logic functions necessary for computation.
In this case, the functions represented were the AND and OR gates:


In order to match the digital quality of the logic functions, the ``0'' and ``1'' states for the input voltages were represented by turning a 5 volt DC source either on (``1'') or off (``0'').
The output voltage states were slightly different, since the architecture of the circuits prevent the output from always reaching a perfect 0 or 5 volts.
Instead, the ``0'' state was defined as any output voltage lower than 1 volt, and the ``1'' state as any voltage higher than 4 volts.
The two functions represented in this lab were the AND and OR gates.


\begin{table}[H]
\centering{}
\caption{AND Gate Truth Table}
\begin{tabular}{@{}ccc@{}}
\toprule
\textbf{Input 1} & \multicolumn{1}{l}{\textbf{Input 2}} & \multicolumn{1}{l}{\textbf{Output}} \\ \midrule
1                & 1                                    & 1                                   \\
1                & 0                                    & 0                                   \\
0                & 1                                    & 0                                   \\
0                & 0                                    & 0                                   \\ \bottomrule
\end{tabular}
\end{table}

\begin{table}[h]
\centering{}
\caption{OR Gate Truth Table}
\begin{tabular}{@{}ccc@{}}
\toprule
\textbf{Input 1} & \multicolumn{1}{l}{\textbf{Input 2}} & \multicolumn{1}{l}{\textbf{Output}} \\ \midrule
1                & 1                                    & 1                                   \\
1                & 0                                    & 1                                   \\
0                & 1                                    & 1                                   \\
0                & 0                                    & 0                                   \\ \bottomrule
\end{tabular}
\end{table}

In both circuits, the behavior of the output voltage matched the output behavior of their corresponding logic functions.
Every output voltage larger than 4 volts corresponded to a truth table output of ``1``, and every output voltage less than 1 volt corresponded to a truth table value of 0.
These results verify the ability of basic circuits to model logic gates using diodes, and gives a glimpse into the kinds of circuitry used in modern computers.

It should be noted that these results only matched when the input voltages were replaced with closed circuits when in the ''0`` (off) state.
For example, if voltage A in the AND gate was turned off (disconnected from the circuit), the positive terminal of A had to be shorted with the negative ground, or else the voltage developed in the output would never discharge and remain charged at 1.

This suggests that the power sources used for the input voltages in computers must have some mechanism or structure that allow them to short (or at least keep from remaining open) when in the off state.


\appendices{}

%\bibliographystyle{plain}
%\bibliography{physbib}

\end{document}