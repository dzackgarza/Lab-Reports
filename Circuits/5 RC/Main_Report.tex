\documentclass[twocolumn,english]{IEEEtran}
\usepackage[T1]{fontenc}
\usepackage{babel}
\usepackage{amsthm}
\usepackage{amsmath}
\usepackage{graphicx}
\usepackage[unicode=true,
 bookmarks=true,bookmarksnumbered=true,bookmarksopen=true,bookmarksopenlevel=1,
 breaklinks=false,pdfborder={0 0 0},backref=false,colorlinks=false]
 {hyperref}
\usepackage{bm}
\usepackage{amsmath}
\usepackage{amssymb}
\usepackage{natbib}
\usepackage{array}
\usepackage{calc}
\newcolumntype{W}{>{\centering\arraybackslash}m{25mm}}
\newcolumntype{L}{>{\centering\arraybackslash}m{15mm}}


\hypersetup{
 pdftitle=  {Lab 5: RC Circuits},
 pdfauthor= {Adam Olson, Nick Lonsdale, Zack Garza},
 pdfpagelayout=OneColumn, pdfnewwindow=true, pdfstartview=XYZ, plainpages=false}

\makeatletter


%%%%%%%%%%%%%%%%%%%%%%%%%%%%%% Textclass specific LaTeX commands.
 % protect \markboth against an old bug reintroduced in babel >= 3.8g
 \let\oldforeign@language\foreign@language
 \DeclareRobustCommand{\foreign@language}[1]{%
   \lowercase{\oldforeign@language{#1}}}
\theoremstyle{plain}
\newtheorem{thm}{\protect\theoremname}
\theoremstyle{plain}
\newtheorem{lem}[thm]{\protect\lemmaname}

%%%%%%%%%%%%%%%%%%%%%%%%%%%%%% User specified LaTeX commands.
% for subfigures/subtables
\ifCLASSOPTIONcompsoc
\usepackage[caption=false,font=normalsize,labelfont=sf,textfont=sf]{subfig}
\else
\usepackage[caption=false,font=footnotesize]{subfig}
\fi

\makeatother
\providecommand{\lemmaname}{Lemma}
\providecommand{\theoremname}{Theorem}
\setcounter{topnumber}{2}
\setcounter{bottomnumber}{2}
\setcounter{totalnumber}{4}
\renewcommand{\topfraction}{0.85}
\renewcommand{\bottomfraction}{0.85}
\renewcommand{\textfraction}{0.15}
\renewcommand{\floatpagefraction}{0.7}
\usepackage{float}
\begin{document}

\title{Lab 5: RC Circuits}


\author{Adam Olson, Nick Lonsdale, Zack Garza}


\IEEEspecialpapernotice
{Engineering 17L \\
Effective Date of Report: April 22, 2014}


\markboth{Lab 5: RC Circuits}{Zack Garza}
\maketitle
\begin{abstract}
The purpose of this lab is to explore some common applications of RC Circuits -- in particular, circuits that effectively perform mathematical operations and circuits that selectively filter input signal frequencies.
\end{abstract}
\tableofcontents

\section{Introduction}
\IEEEPARstart{T}{his} is an introduction.

\section{Theory}

\section{Equipment List}
\section{Methodology}
	\subsection{Differentiator}
	\subsection{Integrator}
	\subsection{Low-Pass Filter}
	\subsection{High-Pass Filter}
	\subsection{Band-Pass Filter}

\section{Results}
	\subsection{Differentiator}
	\subsection{Integrator}
	\subsection{Low-Pass Filter}
	\subsection{High-Pass Filter}
	\subsection{Band-Pass Filter}

\appendices{}

\section{Derivations}
\section{Equipment Photographs}
\section{Circuit Diagrams}
\section{Circuit Photographs}


%\bibliographystyle{plain}
%\bibliography{physbib}

\end{document}