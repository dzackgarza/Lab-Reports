\documentclass[twocolumn,english]{IEEEtran}
\usepackage[T1]{fontenc}
\usepackage{babel}
\usepackage{amsthm}
\usepackage{amsmath}
\usepackage{graphicx}
\usepackage[unicode=true,
 bookmarks=true,bookmarksnumbered=true,bookmarksopen=true,bookmarksopenlevel=1,
 breaklinks=false,pdfborder={0 0 0},backref=false,colorlinks=false]
 {hyperref}
\usepackage{bm}
\usepackage{amsmath}
\usepackage{amssymb}
\usepackage{natbib}
\usepackage{array}
\usepackage{calc}
\newcolumntype{W}{>{\centering\arraybackslash}m{25mm}}
\newcolumntype{L}{>{\centering\arraybackslash}m{15mm}}


\hypersetup{
 pdftitle=  {Lab 3: Digital Oscilloscope Memo},
 pdfauthor= {Zack Garza},
 pdfpagelayout=OneColumn, pdfnewwindow=true, pdfstartview=XYZ, plainpages=false}

\makeatletter


%%%%%%%%%%%%%%%%%%%%%%%%%%%%%% Textclass specific LaTeX commands.
 % protect \markboth against an old bug reintroduced in babel >= 3.8g
 \let\oldforeign@language\foreign@language
 \DeclareRobustCommand{\foreign@language}[1]{%
   \lowercase{\oldforeign@language{#1}}}
\theoremstyle{plain}
\newtheorem{thm}{\protect\theoremname}
\theoremstyle{plain}
\newtheorem{lem}[thm]{\protect\lemmaname}

%%%%%%%%%%%%%%%%%%%%%%%%%%%%%% User specified LaTeX commands.
% for subfigures/subtables
\ifCLASSOPTIONcompsoc
\usepackage[caption=false,font=normalsize,labelfont=sf,textfont=sf]{subfig}
\else
\usepackage[caption=false,font=footnotesize]{subfig}
\fi

\makeatother
\providecommand{\lemmaname}{Lemma}
\providecommand{\theoremname}{Theorem}
\setcounter{topnumber}{2}
\setcounter{bottomnumber}{2}
\setcounter{totalnumber}{4}
\renewcommand{\topfraction}{0.85}
\renewcommand{\bottomfraction}{0.85}
\renewcommand{\textfraction}{0.15}
\renewcommand{\floatpagefraction}{0.7}
\usepackage{float}
\begin{document}

\title{Digital Oscilloscope Memo}


\author{Zack Garza}


\IEEEspecialpapernotice
{Engineering 17L\\
Effective Date of Report: March 18, 2014}


\markboth{Digital Oscilloscope Memo}{Zack Garza}
\maketitle
\begin{abstract}
\IEEEPARstart{T}{he} purpose of this memo is to discuss several parameters on the digital oscilloscope and how they effect readings and measurements. The oscilloscope will then be used to simultaneously measure two channels of information from an LRC circuit to observe the circuit's response to a range of frequencies.
\end{abstract}

\tableofcontents

\section{Theory}

\section{Methodology}
\begin{enumerate}
 \item The signal generator was initialized with the frequency given in Table~\ref{tb:given_parameters} and connected to channel 1 of the oscilloscope.
 \item The self-calibration function on the oscilloscope was activated. The probes were then calibrated as well.
 \item The circuit shown in~\ref{fig:circuit} was constructed, with the initial values described in Table~\ref{tb:given_parameters}. Several settings were modified, and the changes and effects were recorded.
 \item Using the auto-measuring display, the data in Table~\ref{tb:data}.
\end{enumerate}

\begin{table}[h]
\centering{}
\caption{Given Lab Parameters}
\label{tb:given_parameters}
\begin{tabular}{|c|c|}
\hline
\textbf{Wave Type}                                                       & Sine      	\\ \hline
\textbf{Frequency}                                                       & 100 kHz   	\\ \hline
\textbf{\begin{tabular}[c]{@{}c@{}}Peak to Peak \\ Voltage\end{tabular}} & 12 V      	\\ \hline
\textbf{Resistance}                                                      & 12k$\Omega$ 	\\ \hline
\textbf{Inductance}                                                      & 17 mH     	\\ \hline
\textbf{Capacitance}                                                     & 220$\mu$F  	\\ \hline
\textbf{DC Offset}							 & 0 V		\\ \hline
\end{tabular}
\end{table}

\section{Data}
\subsection{Parameter Variations}

\textbf{AC / DC Coupling: }

\textbf{1x / 10x Probe: }

\textbf{Invert On / Off}

\subsection{Auto Measuring}
\begin{align*}
 &\text{RMS Voltage: \underline{$4.17$ V}}	&\text{Peak to Peak Amplitude: \underline{$11.8$ V}} 	\\
 &\text{Period: \underline{$1$ ms}} 		&\text{Frequency: \underline{$(100 \pm 0.2)$ kHz}}	\\
\end{align*}

\subsection{Triggering}
\begin{align*}
 \text{Triggering Type:}&	&\text{\underline{Edge}}\\
 \text{Triggering Source:}&	&\text{\underline{Ch. 2}}\\
 \text{Triggering Mode:}&	&\text{\underline{Auto}}
\end{align*}

\subsection{Dual Signals}
The component from the circuit chosen for measurement on Channel 2 was the $17$ mH capacitor. In order to exhibit a discernible response in the capacitor, the frequency was adjusted. The following measurements were taken with a $28.850$ Hz sine wave.

\noindent Peak to Peak measurements for both channels:
\begin{align*}
 &\text{Ch.1 Peak to Peak: }	&\text{\underline{$11.4$ V}}	\\
 &\text{Ch.2 Peak to Peak: }	&\text{\underline{$9.76$ mV}}
\end{align*}
\noindent Additional measurements taken:
\begin{align*}
 T_S \text{(Peak to Peak, Source): }	&\text{\underline{34.40 ms}} \\
 T_C \text{(Peak to Peak, Capacitor): }	&\text{\underline{34.80 ms}}
\end{align*}

\noindent Phase shift between both channels:
\begin{align*}
 &\Delta t = 8.400\text{ms}	&\theta = 1.534\text{ rad} = 87.9^{\circ}
\end{align*}
Where $\theta$ is given by  $2\pi\frac{\Delta t}{T}$ or $2\pi f \Delta t$ in radians or $360\frac{\Delta t}{T}$ in degrees, $\Delta t$ is the distance in the time domain between two identical points on the adjacent waves, $T$ is the period (time in seconds it takes for a single wave to repeat), and $f$ is the frequency of the generator..

\section{Analysis}
Inductive Reactance: $X_L = 2\pi f L$.

Capacitive Reactance: $X_C = 1/2\pi f C$.

Resonant Frequency: $f_0 = 1/2\pi\sqrt{LC}$
%Comment on phase shift and voltage level.
\appendices{}

\section{Derivation}\label{append:deriv}


%\bibliographystyle{plain}
%\bibliography{physbib}

\end{document}