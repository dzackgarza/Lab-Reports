\documentclass[twocolumn,english]{IEEEtran}
\usepackage[T1]{fontenc}
\usepackage{babel}
\usepackage{amsthm}
\usepackage{amsmath}
\usepackage{graphicx}
\usepackage[unicode=true,
 bookmarks=true,bookmarksnumbered=true,bookmarksopen=true,bookmarksopenlevel=1,
 breaklinks=false,pdfborder={0 0 0},backref=false,colorlinks=false]
 {hyperref}
\usepackage{bm}
\usepackage{amsmath}
\usepackage{amssymb}
\usepackage{natbib}
\usepackage{array}
\usepackage{calc}
\newcommand{\vb}[1]{\mathbf{#1}}		%Bold vector
\newcolumntype{W}{>{\centering\arraybackslash}m{25mm}}
\newcolumntype{L}{>{\centering\arraybackslash}m{15mm}}
\usepackage{booktabs}
\setlength{\parindent}{0pt}

%%%%%%%%%%%%%%%%%%%%%%%%%%%%%%%%%%%%%%%%%%%%%%%%%%%%%%%%%%%%%%%%%%%%%%%%%%%%%%% Variables
\newcommand{\thetitle}{RLC Circuits}
\newcommand{\theauthors}{Zack Garza}
\newcommand{\theclass}{Physics 210L}
%%%%%%%%%%%%%%%%%%%%%%%%%%%%%%%%%%%%%%%%%%%%%%%%%%%%%%%%%%%%%%%%%%%%%%%%%%%%%%%%%%%%%%%%%%

\hypersetup{
 pdftitle=  {\thetitle},
 pdfauthor= {\theauthors},
 pdfpagelayout=OneColumn, pdfnewwindow=true, pdfstartview=XYZ, plainpages=false}

\makeatletter


%%%%%%%%%%%%%%%%%%%%%%%%%%%%%% Textclass specific LaTeX commands.
 % protect \markboth against an old bug reintroduced in babel >= 3.8g
 \let\oldforeign@language\foreign@language
 \DeclareRobustCommand{\foreign@language}[1]{%
   \lowercase{\oldforeign@language{#1}}}
\theoremstyle{plain}
\newtheorem{thm}{\protect\theoremname}
\theoremstyle{plain}
\newtheorem{lem}[thm]{\protect\lemmaname}

%%%%%%%%%%%%%%%%%%%%%%%%%%%%%% User specified LaTeX commands.
% for subfigures/subtables
\ifCLASSOPTIONcompsoc
\usepackage[caption=false,font=normalsize,labelfont=sf,textfont=sf]{subfig}
\else
\usepackage[caption=false,font=footnotesize]{subfig}
\fi

\makeatother
\providecommand{\lemmaname}{Lemma}
\providecommand{\theoremname}{Theorem}
\setcounter{topnumber}{2}
\setcounter{bottomnumber}{2}
\setcounter{totalnumber}{4}
\renewcommand{\topfraction}{0.85}
\renewcommand{\bottomfraction}{0.85}
\renewcommand{\textfraction}{0.15}
\renewcommand{\floatpagefraction}{0.7}
\usepackage{float}
\begin{document}

\title{\thetitle}
\author{\theauthors}
\IEEEspecialpapernotice
{\theclass \\ Effective Date of Report: \today }
\markboth{\thetitle}{\theauthors}
\maketitle

\begin{abstract}
\IEEEPARstart{T}{he} purpose of this experiment...
\end{abstract}

\tableofcontents

\section{Theory}

The following expressions were used in the analysis of this circuit:
\begin{align*}
	\omega &= 2\pi f \\
	\omega_0 &= \sqrt{\frac{1}{LC}} \Rightarrow f_0 = \frac{1}{2\pi\sqrt{LC}}\\
	X_L &= \frac{V_L}{I} \\
	X_C &= \frac{V_C}{I} \\
	Z &= \frac{V_s}{I} \\
	L &= \frac{X_L}{2\pi f} \\
	C &= \frac{1}{2\pi f X_C} \\
	\theta &= 360^{\circ} f\Delta t \\
	P &= I_{\text{rms}} V_{\text{rms}} \cos\theta \\
	Q &= \frac{\omega_0 L}{R} \\
	Q &= \frac{f_0}{\Delta f} = \frac{\omega_0}{\Delta \omega}
\end{align*}

$\Delta f$ is the width of the resonance peak between the points where $V = \frac{1}{\sqrt{2}}V_{\text{max}}$

\section{Methodology}
\begin{enumerate}
	\item A simple RLC series circuit was constructed. DMMs were wired to measure $\Delta V_L$, and $\Delta V_C$, the voltages of the inductor $L$ and the capacitor $C$ respectively.
	\item An ammeter was wired in series with the previous elements to measure $I$, and an oscilloscope was wired to measure $\Delta V_s$ and $\Delta V_R$, the voltages across the source and the resistor $R$ respectively.
	\item The resistance of $R$ was measured in order to determine phase difference between the source voltage $V_s$ and the current $I$. $\Delta t$ between these two signals was recorded.
	\item The potential difference of the supply was set to a constant 2.0 V (rms), and the ammeter was set to the 430 mA scale.
	\item The expected resonance frequency $f_0$ was calculated, and a range of frequencies symmetric about $f_0$ were chosen for measurement.
	\item For each frequency $f$, the following quantities were measured:
	\begin{enumerate}
		\item $I_{\text{rms}}$
		\item $V_L$
		\item $V_C$
	\end{enumerate}
	Extra data points were taken at frequencies approaching $f_0$.
	\item The data was then placed into a spreadsheet to calculate various variables, which were used in the analysis section.
\end{enumerate}

\section{Results}
\subsection{Average Inductance/Capacitance Values}

\underline{$\bar L = n\pm d$}

\underline{$\bar C = n\pm d$}

\hrulefill

\subsection{Impedance vs. Frequency}

%Plot of {XL, XC, Z} vs. f
%Determine Zmin and f resonant
%
% \begin{figure}[h!]
% 	\begin{centering}
% 	\begin{center}
% 	\includegraphics[width=\linewidth]{./??????.png}
% 	\caption{Plot of $X_L, X_C, $ and $Z$ vs. $f$. }
% 	\label{fig:xf_graph1}
% 	\end{center}
% 	\par\end{centering}
% \end{figure}

\underline{$Z_{\text{min}} = n \Omega$}

\underline{$f_{\text{res,meas}} = n$ Hz} \\ \\

\textit{Why is $R_{\text{meas}}$ not the same as the value of the resistor used in the circuit?}

\hrulefill

\subsection{Current vs. Frequency}

The data was fitted to the function
\begin{equation}
	I_{\text{rms}}
	= \frac
	{V_{\text{rms}}\omega}
	{\sqrt{(R\omega)^2 +L^2(\omega^2-\omega_0^2)^2}}
\end{equation}

% \begin{figure}[h!]
% 	\begin{centering}
% 	\begin{center}
% 	\includegraphics[width=\linewidth]{./IvsFreq.png}
% 	\caption{Plot of $I_{\text{rms}} vs. $f$}
% 	\label{fig:IvsFreq}
% 	\end{center}
% 	\par\end{centering}
% \end{figure}

\underline{$R_{\text{fit}} = n \Omega$}

\underline{$f_{\text{res, fit}} = n$ Hz}

\hrulefill

\subsection{Difference between measured and fitted values}

\textit{What is the percent difference between $R$ (measured) and $R$ (fit)? Is $R$ (measured) within the uncertainty of $R$ (fit)?}

\underline{Percent Difference: $x\%$}

\hrulefill

\subsection{Theoretical Resonant Frequency}
The resonant frequency, calculated from the mean values of $L$ and $C$ from step 1 is given by
\begin{align}
	f_{\text{res,theor}} &= \frac{1}{2\pi\sqrt{L_{\text{avg}} C_{\text{avg}}}}. \\
	\Rightarrow f_{\text{res,theor}} &= n Hz \notag
\end{align}

Comparing this to the values determined in Steps 2 and 3 above, we have

\underline{\% Difference, Step 2: $x\%$}

\underline{\% Difference, Step 3: $x\%$}

\hrulefill

\subsection{Voltage vs Frequency}

% \begin{figure}[h!]
% 	\begin{centering}
% 	\begin{center}
% 	\includegraphics[width=\linewidth]{./VvsFreq.png}
% 	\caption{Plot of $V_L$ and $V_C$ vs $f$, extrapolated to zero frequency.}
% 	\label{fig:VvsFreq}
% 	\end{center}
% 	\par\end{centering}
% \end{figure}

\textit{What does this graph tell you?}

\hrulefill

\subsection{Phase Angle vs. Frequency}

% \begin{figure}[h!]
% 	\begin{centering}
% 	\begin{center}
% 	\includegraphics[width=\linewidth]{./PhasevsFreq.png}
% 	\caption{Plot of phase angle ($360^{\circ}f\Delta t$) vs. $f$, with $I-V$ relationship denoted.}
% 	\label{fig:PhasevsFreq}
% 	\end{center}
% 	\par\end{centering}
% \end{figure}

\textit{Does this curve agree with theory? Explain.}

\hrulefill

\subsection{Power vs. Frequency}

% \begin{figure}[h!]
% 	\begin{centering}
% 	\begin{center}
% 	\includegraphics[width=\linewidth]{./PowervsFreq.png}
% 	\caption{Plot of $P$ vs $f$, used to determine the full-width and half-max in order to calculate $Q$.}
% 	\label{fig:PowervsFreq}
% 	\end{center}
% 	\par\end{centering}
% \end{figure}

\underline{$\Delta f$ = x Hz}

\hrulefill

\subsection{Quality Factor}
The calculated quality factor is given by
\begin{equation}
	Q_{\text{calc}} = \frac{f_{\text{max, meas}}}{\Delta f}
\end{equation}

Calculated Quality Factor: \underline{$Q_{\text{calc}} = x\pm d$}
%1Hz uncertainty in f and delta f

The theoretical Quality factor is given by
\begin{equation}
	Q = \frac{\omega_0 L}{R_{\text{meas}}},
\end{equation}
where $\omega_0$ is given by $2\pi f_{\text{res,meas}}$ from Step 2 and $L$ is $L_{\text{avg}}$ from Step 1.

Theoretical Quality Factor: \underline{$Q = x\pm d$}

\hrulefill

\section{Analysis and Conclusion}

%\appendices{}
%\bibliographystyle{plain}
%\bibliography{physbib}

\end{document}