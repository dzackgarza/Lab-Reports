\documentclass[twocolumn,english]{IEEEtran}
\usepackage[T1]{fontenc}
\usepackage{babel}
\usepackage{amsthm}
\usepackage{amsmath}
\usepackage{graphicx}
\usepackage[unicode=true,
 bookmarks=true,bookmarksnumbered=true,bookmarksopen=true,bookmarksopenlevel=1,
 breaklinks=false,pdfborder={0 0 0},backref=false,colorlinks=false]
 {hyperref}
\usepackage{bm}
\usepackage{amsmath}
\usepackage{amssymb}
\usepackage{natbib}
\usepackage{array}
\usepackage{calc}
\newcolumntype{W}{>{\centering\arraybackslash}m{25mm}}
\newcolumntype{L}{>{\centering\arraybackslash}m{15mm}}


\hypersetup{
 pdftitle=  {Lab 5: Circuit Basics},
 pdfauthor= {Zack Garza},
 pdfpagelayout=OneColumn, pdfnewwindow=true, pdfstartview=XYZ, plainpages=false}

\makeatletter


%%%%%%%%%%%%%%%%%%%%%%%%%%%%%% Textclass specific LaTeX commands.
 % protect \markboth against an old bug reintroduced in babel >= 3.8g
 \let\oldforeign@language\foreign@language
 \DeclareRobustCommand{\foreign@language}[1]{%
   \lowercase{\oldforeign@language{#1}}}
\theoremstyle{plain}
\newtheorem{thm}{\protect\theoremname}
\theoremstyle{plain}
\newtheorem{lem}[thm]{\protect\lemmaname}

%%%%%%%%%%%%%%%%%%%%%%%%%%%%%% User specified LaTeX commands.
% for subfigures/subtables
\ifCLASSOPTIONcompsoc
\usepackage[caption=false,font=normalsize,labelfont=sf,textfont=sf]{subfig}
\else
\usepackage[caption=false,font=footnotesize]{subfig}
\fi

\makeatother
\providecommand{\lemmaname}{Lemma}
\providecommand{\theoremname}{Theorem}
\setcounter{topnumber}{2}
\setcounter{bottomnumber}{2}
\setcounter{totalnumber}{4}
\renewcommand{\topfraction}{0.85}
\renewcommand{\bottomfraction}{0.85}
\renewcommand{\textfraction}{0.15}
\renewcommand{\floatpagefraction}{0.7}
\usepackage{float}
\begin{document}

\title{Circuit Basics}


\author{Zack Garza}


\IEEEspecialpapernotice
{Physics 210L \\
Effective Date of Report: March 18, 2014}


\markboth{Circuit Basics}{Zack Garza}
\maketitle
\begin{abstract}
Placeholder
\end{abstract}
\tableofcontents

\section{Introduction}
\IEEEPARstart{T}{his} experiment will investigate an important property common to all electrical circuits known as resistance. The relationship between resistance, current, and potential difference (Ohm's Law) will be examined and utilized.

\begin{centering}
 \textbf{Objectives}
\end{centering}
The experimental objectives of this three part activity are:
\begin{enumerate}
 \item Part 1: To measure the resistance of test leads.
 \item Part 2: To learn how to wire a circuit on a Proto-board and to study the effects of ammeters and voltmeters in a circuit.
 \item Part 3: To learn how to wire series and parallel circuits.
\end{enumerate}


\section{Theory}

\section{Methodology}
\textbf{Part 1}
\begin{enumerate}
 \item X leads were wired together in series.
 \item The resistance of the entire series of leads ($R_eq$) was measured with a multimeter and recorded.
 \item The number of leads used ($n_L$) was recorded.
 \item The length of series of leads was measured with a meter stick and recorded.
\end{enumerate}

\textbf{Part 2}
\begin{enumerate}
 \item A digital multimeter was used to measure the resistance between points in the protoboard in order to determine its internal wiring.
 \item The circuit shown in Figure~\ref{fig:part2_circuit} was constructed with $R_1=n$ and $\varepsilon = e$, both in series with the DMM in ammeter mode.
 \item The current $I_1$ through resistor $R_1$ was measured and recorded.
 \item A DMM in voltmeter mode was wired across $R_1$ and the change in ammeter's reading was recorded as $I_2$.
 \item The voltmeter was replaced with an analog voltmeter on the 1.5 V scale, and the change in the ammeter's reading was recorded as $I_3$.
  %Is this expected?
 \item The analog voltmeter was disconnected, and its resistance was measured while on the 1.5 V scale with a DMM and recorded as $V_{\text{analog}}$.
 \item The measurement of resistance was repeated for the digital voltmeter and recorded as $V_{\text{digital}}$.
 \item The circuit was reset to the state shown in ~\ref{fig:part2_circuit}.
 \item A DMM in voltmeter mode was wired across the resistor $R_1$ and the battery.
 \item The voltage across $R_1$ was recorded as $V_1$, and the voltage across the battery was recorded as $\epsilon_1$.
 \item The multimeter was left on the scale used for the previous measurement, and its resistance was measured. The scale setting and resistance were recorded.
 \item The previous measurements were repeated using a different ammeter scale. The voltage was recorded as $V_2$ and $\epsilon_2$. The scale and resistance of the ammeter were also measured and recorded.
\end{enumerate}

\textbf{Part 3}

\begin{enumerate}
 \item \textbf{Series Resistor Combinations} \begin{enumerate}
        \item Measured the resistance of the three resistors provided.
        \item A series circuit was built with these resistors on a protoboard, and the equivalent resistance was measured with an ohmmeter.
        \item A power supply was chosen to connect to the series circuit.The maximum current the circuit could carry was calculated, given that the resistors had a power limitation of 1/4 W. The maximum current and voltage were recorded.
        \item The power supply was wired to the circuit, and the voltages across the power supply and across each resistor were measured.
       \end{enumerate}
  \item \textbf{Parallel Resistor Combinations} \begin{enumerate}
         \item Two of the resistors were chosen, and a parallel combination was constructed on the protoboard. The equivalent resistance of the circuit was measured and recorded.
         \item A power supply was selected, and the maximum theoretical voltage and current were calculated and recorded.
         \item The circuit was powered by the supply, and the currents through the power supply and through each of the resistors were measured.
        \end{enumerate}
\end{enumerate}

\section{Data}
  \subsection*{\textbf{Part 1}}
  \begin{align*}
   &\text{Number of Test Leads Used:}		&\text{\underline{Stuff}} \\
   &\text{Total Resistance of Test Leads: }	&\text{\underline{Stuff}}
  \end{align*}

  \subsection*{\textbf{Part 2}}
  Current through $R_1$:
  \begin{align*}
   &\text{$I_1$ = $n$A}
  \end{align*}

  Current through $R_1$ (with digital voltmeter in circuit):
  \begin{align*}
   &I_2  &n A \\
   &\text{Deviation from $I_1$:} &n A
  \end{align*}

  Current through $R_1$ (with analog voltmeter in circuit):
  \begin{align*}
   &I_3: & n A \\
   &\text{Deviation from $I_1$:} &n A
  \end{align*}

  Meter Resistance Measurements:
  \begin{align*}
   R_{\text{analog-voltmeter}} 	&= n \\
   R_{\text{digital-voltmeter}}	&= n
  \end{align*}

  Voltage Measurements with DMM:
  \begin{align*}
   &\text{Resistor Voltage:} & n \Omega \\
   &\text{Battery Voltage:}  & n \Omega
  \end{align*}

  Ammeter Resistance for above measurement:
  \begin{align*}
   &\text{Scale Used: $n$ } &R_\text{digital-ammeter} = n A \\
  \end{align*}

  Voltage Measurements with DMM (with ammeter at different scale):
  \begin{align*}
   &\text{Resistor Voltage:} & n \Omega \\
   &\text{Battery Voltage:}  & n \Omega
  \end{align*}

  Ammeter Resistance for previous measurement:
  \begin{align*}
   &\text{Scale Used: $n$ } &R_\text{digital-ammeter} = n A \\
  \end{align*}

  \subsection*{\textbf{Part 2}}
  \textbf{Series}
  \begin{align*}
   &\text{Maximum Current = $n$}	&\text{Maximum Voltage}=n V 	\\
   &\text{Measured Resistance Values: }	R_1=n 	&R_2=n 	&R_3=n 		\\
   &\text{Series Equivalent Resistance: } R = n 			\\
   &V_{\text{Supply}}=n		 V_1=n		&V_2 = nV	&V_3=nV	\\
  \end{align*}

  \textbf{Parallel}
  \begin{align*}
   &\text{Maximum Voltage} = n V			\\
   &\text{Measured Resistance: }	&R_1 = 	&R_2 =	\\
   &\text{Parallel Equivalent Resistance: } R =		\\
   &V_{\text{Supply}} = nV				\\
   &I_S = n V	&I_1 = n V	&I_2 = n V
  \end{align*}

\section{Analysis and Results}
\textbf{Part 1}

\textit{Calculate the average resistance of a test lead and describe the method used.}


\textbf{Part 2}

\textit{Explain the observations made in items 3 and 4. What is the most important characteristic of a voltmeter, and why?}

\textit{Explain any discrepancies in item 6 with voltage measured across the resistor using the multimeter. Does the ammeter affect the circuit? If so, how?}

\textbf{Part 3}
\begin{enumerate}
 \item Series: \begin{enumerate}
               \item Calculate the series equivalent resistance and compare it (with percent difference) to the measured value.
               \item Verify Kirchoff's Voltage Rule by comparing the sum of the voltage drops across each of the resistors with the measured terminal voltage $V_T$.
               \item Verify the voltage divider expression for one of the resistors, $V_{Ri} = \Delta V_S R_i / \sum R_i$
              \end{enumerate}
  \item Parallel: \begin{enumerate}
                   \item Calculate the parallel equivalent resistance and compare it (with percent difference) to the measured value.
                   \item Verify Kirchoff's Current Rule by comparing the sum of the currents through the resistors with the measured current $I_s$.
                   \item Verify the current divider expression by calculating the current through one resistor and comparing it with the measured value, $I_{Ri} = I_S R_{eq} / R_i$.
                   %TODO Fix equations.
                  \end{enumerate}


\end{enumerate}

\appendices{}

\section{Derivation}\label{append:deriv}


%\bibliographystyle{plain}
%\bibliography{physbib}

\end{document}