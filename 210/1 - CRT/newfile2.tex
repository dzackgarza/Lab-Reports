\documentclass[twocolumn,english,amsmath,amssymb]{IEEEtran}
\usepackage[T1]{fontenc}
\usepackage{babel}
\usepackage{amsthm}
\usepackage{graphicx}
\usepackage[unicode=true,
 bookmarks=true,bookmarksnumbered=true,bookmarksopen=true,bookmarksopenlevel=1,
 breaklinks=false,pdfborder={0 0 0},backref=false,colorlinks=false]
 {hyperref}
\hypersetup{
 pdftitle=  {Lab 2: The Cathode Ray Tube},
 pdfauthor= {Zack Garza},
 pdfpagelayout=OneColumn, pdfnewwindow=true, pdfstartview=XYZ, plainpages=false}

\makeatletter

%%%%%%%%%%%%%%%%%%%%%%%%%%%%%% LyX specific LaTeX commands.
\DeclareRobustCommand*{\lyxarrow}{%
\@ifstar
{\leavevmode\,$\triangleleft$\,\allowbreak}
{\leavevmode\,$\triangleright$\,\allowbreak}}
%% Because html converters don't know tabularnewline
\providecommand{\tabularnewline}{\\}
%% A simple dot to overcome graphicx limitations
\newcommand{\lyxdot}{.}


%%%%%%%%%%%%%%%%%%%%%%%%%%%%%% Textclass specific LaTeX commands.
 % protect \markboth against an old bug reintroduced in babel >= 3.8g
 \let\oldforeign@language\foreign@language
 \DeclareRobustCommand{\foreign@language}[1]{%
   \lowercase{\oldforeign@language{#1}}}
\theoremstyle{plain}
\newtheorem{thm}{\protect\theoremname}
\theoremstyle{plain}
\newtheorem{lem}[thm]{\protect\lemmaname}

%%%%%%%%%%%%%%%%%%%%%%%%%%%%%% User specified LaTeX commands.
% for subfigures/subtables
\ifCLASSOPTIONcompsoc
\usepackage[caption=false,font=normalsize,labelfont=sf,textfont=sf]{subfig}
\else
\usepackage[caption=false,font=footnotesize]{subfig}
\fi

\makeatother

\providecommand{\lemmaname}{Lemma}
\providecommand{\theoremname}{Theorem}
\usepackage{bm}
\usepackage{amsmath}
\usepackage{amssymb}
\usepackage{natbib}
\bibliographystyle{plainnat}
\bibliography{phys2}

\begin{document}

\title{Lab 2: The Cathode Ray Tube}


\author{Zack Garza}


\IEEEspecialpapernotice
{Physics 215L \\
Prepared for: Professor Calabrese \\
Effective Date of Report: February 25, 2014}


\markboth{Lab 2: The Cathode Ray Tube}{Zack Garza}
\maketitle
\begin{abstract}
It should briefly mention the motivation, the method and
most important, the quantitative result with errors.
\end{abstract}

\section{Introduction}



\IEEEPARstart{T}{he} cathode ray tube is the central component
of a number of familiar devices including the television set, the
personal computer and the oscilloscope. In this experiment, you will
explore the characteristics of a CRT using the Demonstration Unit
$\#24-3010$. This will result in the verification of the effective ratio
of length to spacing $(l/d)$ of the deflecting plates and to an understanding
of how the anodes focus the electron beam. The concepts of electric
field and electric potential will be reinforced and the information
gained will serve as valuable preparation for Experiment 7.


\section{Theory}
Referring to the accompanying schematic of the CRT and using the information provided by your instructor, derive an equation for the vertical deflection $y$ of the electron beam in terms of $L$, $l$, $d$, $V_1$, \& $V_2$. All of the important quantities you will use are listed below: \\

\noindent $y=$ Beam deflection at the screen \\
$q=$ Electron Charge \\
$E=$ Electric field strength between plates (uniform) \\
$F=$ deflecting force (on q) $=qE$ \\
$V_1=$ accelerating potential (second anode) \\
$a=$ acceleration charge between plates \\
$t=$ time that q is between plates \\
$m=$ mass of electron \\
$d=$ plate separation = 0.234 cm \\
$l=$ plate length = 3.80 cm \\
$L=$ plate to screen distance = 12.0 cm \\
$V_2=$ deflecting potential $=Ed$ \\

\section{Methodology}

\subsection{Operation of the CRT}
\begin{enumerate}
 \item Turn off the CRT unit ($\#24-3010$), and the Power Supply ($EUW-17$). Set the intensity control to a minimum (full CCW).  Set the power control to a minimum (full CCW). \\

 Now turn on the CRT and the Power Supply. \\

 \textbf{CAUTION!  THE DEFLECTING PLATE TERMINALS AND THE SECOND ANODE ARE AT 700 VOLTS ABOVE GROUND.  DON'T TOUCH THESE LEADS!} \\

 \item
 Adjust the intensity control until a spot is visible on the screen.  The spot should be no more intense than necessary to see it conveniently.  A high intensity can burn the phosphor coating on the inside of the screen.  Carefully measure the position of the spot using the grid on the screen.  This will be your reference position for future spot positions. \\

 \item Deflect the beam at equal voltage intervals ($V_2$) with an appropriate value that will result in no less than ten data points covering a complete screen deflection. Measure the position of the spot at each value of $V_2$.  Record $V_1$ and ordered pairs of $V_2$ and $y$ (to 0.1 mm) in Table 1. \\

 \item When you are finished, turn the intensity and power supply controls to a minimum and turn off all the instruments (CRT unit, power supply, and voltmeters). \\
\end{enumerate}

\subsection{Calculations}
\begin{enumerate}
 \item Graph $y$ (mm) vs $V_2$ (volts).  You should see a linear plot for which a curve fit will yield the slope. The number of decimal places you keep in the slope depends on the standard deviation on the slope of your graph! Calculate the ratio of $l/d$ (calculated), and compare this with $l/d$ (measured).  Copy and paste your graph into this report. Don't forget to show sample calculations. \\

 \item Complete the attached worksheet titled "CRT Electron Focusing". \\

 \item Calculate $V_x$, the horizontal speed of the electron beam and attach this calculation to your report. \\
\end{enumerate}


\newpage
\section{Results}
\centerline{\underline{$ \Delta V_1 = 644 V$}}
\begin{centering}
  \begin{table}[htbp]
  \caption{y vs $\Delta V_2$}
  \centering{}
  \begin{tabular*}{\linewidth}{@{\extracolsep{\fill}} |c|c|c|}
  \hline
  \textbf{Trial} & \textbf{$\Delta V_2$ (volts)} & \textbf{$y$ (mm)} \\ \hline
  1  & 3.048  & 1  \\ \hline
  2  & 3.732  & 2  \\ \hline
  3  & 4.40   & 3  \\ \hline
  4  & 4.96   & 4  \\ \hline
  5  & 5.70   & 5  \\ \hline
  6  & 6.49   & 6  \\ \hline
  7  & 7.07   & 7  \\ \hline
  8  & 7.70   & 8  \\ \hline
  9  & 8.46   & 9  \\ \hline
  10 & 9.06   & 10 \\ \hline
  11 & 9.90   & 11 \\ \hline
  12 & 10.51  & 12 \\ \hline
  13 & 11.19  & 13 \\ \hline
  14 & 11.98  & 14 \\ \hline
  15 & 12.64  & 15 \\ \hline
  \end{tabular*}
  \end{table}
\end{centering}

The data in Table 1 reflects the total deflection of the CRT electron beam, where the beam was initially calibrated to align with $y=0$ on the superimposed grid. The voltage across the the focusing portion was kept constant at $644 V$, while the voltage supplied to the deflecting plates was steadily increased. Data points were taken when the beam aligned with a grid line.


\begin{figure}[htbp]
\begin{centering}
\begin{center}
 \includegraphics[keepaspectratio=true, width=\linewidth]{./CRTLab.png}
 % CRTLab.png: 493x405 pixel, 96dpi, 13.04x10.71 cm, bb=0 0 370 304
\caption{Plot and Linear Regression of Table 1 Data}
\end{center}
\par\end{centering}
\end{figure}

Linear Fit Data:
\begin{align*}
\Delta y & =(1.458 \pm .008)\Delta V_2 - (3.359 \pm .07) \\
R^2 & = 0.999590118089244 \\
RMSE & = 0.0939586386136779
\end{align*}




\section{Conclusions}

bla bla


\appendices{}

\section{Appendix 1: Derivations}

\subsection{Total Vertical Deflection}
In order to calculate the total deflection in terms of the known quantities given in the theory section, the path of the electron can be examined in three stages: focusing, deflection, and the flight from the plates to the screen.
\begin{enumerate}
 \item \textbf{Focusing} \\
 Here, each electron is being accelerated in the $\hat{x}$ direction between two plates at a potential difference of $\Delta V_1$. From the Work-Kinetic Energy Theorem, it can be stated that the initial potential will be equal to the final kinetic energy such that $q \Delta V_1 = \frac{1}{2}m{V_0}^2$. Thus the electron's speed entering part 2 can be expressed as

 \begin{equation} \label{eq:v0}
 {V_0}^2 = \frac{2q\Delta V_1}{m}
 \end{equation}

 \item \textbf{Deflection} \\
 In a similar manner, the electron is now accelerated in the $\mathbf{\hat{y}}$ direction due to the potential difference $\Delta V_2$ between the deflection plates. Neglecting the force of gravity, it follows a parabolic path that obeys kinematic equations. From Newton's Second Law, $F=ma$, where the net force is due to the electric field. Thus $ma = qE$ and

 \begin{equation} \label{eq:accel}
 a = \frac{qE}{m}
 \end{equation}

 The electric field $E$ is a function of the potential difference, where $\Delta V_2 = Ed \Rightarrow E = \frac{\Delta V_2}{d}.$ Substituting this into equation~\ref{eq:accel} yields
 \begin{equation} \label{eq:accel2}
 a = \frac{q\Delta V_2}{md}
 \end{equation}

 Between the deflection plates, the electron will follow a parabolic path that obeys the kinematic equations. Integrating equation~\ref{eq:accel2} twice (wrt time) yields
 \begin{equation} \label{eq:vy1}
  V_{y1} = (\frac{q\Delta V_2}{md})t_1
 \end{equation}
 \begin{equation} \label{eq:y1}
  \Delta y_1 = \frac{1}{2}(\frac{q\Delta V_2}{md}){t_1}^2
 \end{equation}

 Since the acceleration in the $\mathbf{\hat{x}}$ is 0, $\Delta x = V_0 t_1$ and $t_1=\frac{\Delta x}{V_o}.$ Since $\Delta x$ in the deflection plates is the known length $l$, this equation can be expressed as $t_1=\frac{l}{V_o}$. Substituting this into equation~\ref{eq:y1} yields the following expression for the initial deflection.
 \begin{equation} \label{eq:y1final}
  \Delta y_1 = \frac{1}{2}(\frac{q\Delta V_2}{md}){(\frac{l}{V_0})}^2 =\frac{l^2 q\Delta V_2}{2md{V_0}^2}
 \end{equation}

 \item \textbf{Path to Screen} \\
 Neglecting gravity, the electron now has no net force acting upon it, and will continue on the trajectory at which it left the deflection plates. Because the distance from the plates to the screen is known to be $L$, $a_x=0$, and $V_x=V_0$, the kinematic equation of its motion in the $\mathbf{\hat{x}}$ is $\Delta x_2=V_0 t_2$ and thus
 \begin{equation} \label{eq:t2}
 t_2 = \frac{L}{V_0}
 \end{equation}

 Similarly, $a_y=0$, so $\Delta y_2 = (V_i \mathbf{\hat{y}})t$. To obtain $V_i \mathbf{\hat{y}}$, note that this will be equal to the particle's final velocity upon exiting the deflector. Since the time spent travelling through the deflector is known, from equation~\ref{eq:vy1},
 \begin{align*}
  V_{y1} = (\frac{q\Delta V_2}{md})t_1 \Rightarrow V_{iy}= (\frac{q\Delta V_2}{md})(\frac{l}{V_0})
 \end{align*}
  Combining the above expression for $\Delta y_2$ and equation~\ref{eq:t2} yields
  \begin{equation} \label{eq:y2final}
   \Delta y_2 = (\frac{q\Delta V_2}{md})(\frac{l}{V_0})(\frac{L}{V_0}) = \frac{l L q\Delta V_2}{md{V_0}^2}
  \end{equation}

  \item {\textbf{Total Deflection}}\\
  Finally, combining and simplifying equations~\ref{eq:v0}, ~\ref{eq:y1final}, and ~\ref{eq:y2final} yields the expression for the total deflection of the particle in terms of known quantities,
  \begin {align}\label{eq:final}
   \Delta y_T & = \Delta y_1 + \Delta y_2 = \frac{l^2 q\Delta V_2}{2md{V_0}^2} + \frac{l L q\Delta V_2}{md{V_0}^2} \notag \\
   & = (\frac{q \Delta V_2 l^2 + 2lLq\Delta V_2}{2md})(\frac{m}{2q\Delta V_1}) \notag \\
   & = \frac{\Delta V_2 l^2 + 2lL\Delta V_2}{4d\Delta V_1}\notag \\
   & = \Delta V_2 (\frac{l}{d})(\frac{l+2L}{4\Delta V_1})
  \end {align}

\end{enumerate}



\section*{References}
\cite{serway2010physics}

\end{document}