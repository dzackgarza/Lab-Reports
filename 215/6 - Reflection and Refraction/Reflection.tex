\documentclass[twocolumn,english]{IEEEtran}
\usepackage[T1]{fontenc}
\usepackage{babel}
\usepackage{amsthm}
\usepackage{amsmath}
\usepackage{graphicx}
\usepackage[unicode=true,
 bookmarks=true,bookmarksnumbered=true,bookmarksopen=true,bookmarksopenlevel=1,
 breaklinks=false,pdfborder={0 0 0},backref=false,colorlinks=false]
 {hyperref}
\usepackage{bm}
\usepackage{amsmath}
\usepackage{amssymb}
\usepackage{natbib}
\usepackage{siunitx}
\usepackage{array}
\usepackage{calc}
\newcolumntype{W}{>{\centering\arraybackslash}m{25mm}}
\newcolumntype{L}{>{\centering\arraybackslash}m{15mm}}


\hypersetup{
 pdftitle=  {Lab 6: Reflection and Refraction},
 pdfauthor= {Zack Garza},
 pdfpagelayout=OneColumn, pdfnewwindow=true, pdfstartview=XYZ, plainpages=false}

\makeatletter


%%%%%%%%%%%%%%%%%%%%%%%%%%%%%% Textclass specific LaTeX commands.
 % protect \markboth against an old bug reintroduced in babel >= 3.8g
 \let\oldforeign@language\foreign@language
 \DeclareRobustCommand{\foreign@language}[1]{%
   \lowercase{\oldforeign@language{#1}}}
\theoremstyle{plain}
\newtheorem{thm}{\protect\theoremname}
\theoremstyle{plain}
\newtheorem{lem}[thm]{\protect\lemmaname}

%%%%%%%%%%%%%%%%%%%%%%%%%%%%%% User specified LaTeX commands.
% for subfigures/subtables
\ifCLASSOPTIONcompsoc
\usepackage[caption=false,font=normalsize,labelfont=sf,textfont=sf]{subfig}
\else
\usepackage[caption=false,font=footnotesize]{subfig}
\fi

\makeatother
\providecommand{\lemmaname}{Lemma}
\providecommand{\theoremname}{Theorem}
\sisetup{detect-weight=true, detect-family=true}
\setcounter{topnumber}{2}
\setcounter{bottomnumber}{2}
\setcounter{totalnumber}{4}
\renewcommand{\topfraction}{0.85}
\renewcommand{\bottomfraction}{0.85}
\renewcommand{\textfraction}{0.15}
\renewcommand{\floatpagefraction}{0.7}
\usepackage{float}
\begin{document}

\title{Reflection and Refraction}


\author{Zack Garza}


\IEEEspecialpapernotice
{Physics 215L \\
Effective Date of Report: April 2, 2014}


\markboth{Reflection and Refraction}{Zack Garza}
\maketitle
\begin{abstract}
\IEEEPARstart{T}{his} is an abstract.
\end{abstract}
\tableofcontents


\section{Theory}

%State and explain the laws of reflection and refraction (Snell's Law).
%Use diagrams to add clarification.
%Show how the speed of light in a medium (such as glass) relates to the speed of light in a vacuum.

\section{Methodology}

\subsection*{Part 1: Reflection}

\textbf{Introduction to parallax}
\begin{enumerate}
 \item A pencil was held vertically at arm's length. A second pencil was held abut 15 cm closer than the first.
 \item The pencil's were held in place, while the observer moved their head from side to side.
 \item The pencils were moved closer together, and the apparent relative motion was observed.
 \item The pencils were adjusted until there was no apparent relative motion between them. All observations were recorded.
\end{enumerate}

\textbf{Locating an image in a plane mirror}
\begin{enumerate}
 \item A piece of paper was placed on a large piece of cork board, and a plane mirror was placed vertically in the center of the page.
 \item The location of the mirror on the paper was traced with a pencil.
 \item A pin was placed about 10 cm in front of the mirror.
 \item A second pin was placed behind the mirror, and its position was adjusted until no parallax was observable between the second pin and the image of the first pin.
 \item The distance between each pin and the plane of the mirror was measured and recorded.
\end{enumerate}

\textbf{Locating an image using raytracing}



\section{Results}
  \subsection*{\textbf{Part 1: }}



  \textbf{Summary}


\appendices{}

\section{Derivation}\label{append:deriv}


%\bibliographystyle{plain}
%\bibliography{physbib}

\end{document}